%%%%%%%%%%%%%%%%%%%%%%%%%%%%%%%%%%%%%%%%%
% Masters/Doctoral Thesis 
% LaTeX Template
% Version 2.3 (25/3/16)
%
% This template has been downloaded from:
% http://www.LaTeXTemplates.com
%
% Version 2.x major modifications by:
% Vel (vel@latextemplates.com)
%
% This template is based on a template by:
% Steve Gunn (http://users.ecs.soton.ac.uk/srg/softwaretools/document/templates/)
% Sunil Patel (http://www.sunilpatel.co.uk/thesis-template/)
%
% Template license:
% CC BY-NC-SA 3.0 (http://creativecommons.org/licenses/by-nc-sa/3.0/)
%
%%%%%%%%%%%%%%%%%%%%%%%%%%%%%%%%%%%%%%%%%

%----------------------------------------------------------------------------------------
%	PACKAGES AND OTHER DOCUMENT CONFIGURATIONS
%----------------------------------------------------------------------------------------
\PassOptionsToPackage{greek, english}{babel}
\documentclass[
11pt, % The default document font size, options: 10pt, 11pt, 12pt
%oneside, % Two side (alternating margins) for binding by default, uncomment to switch to one side
%chapterinoneline,% Have the chapter title next to the number in one single line
%greek, english, % ngerman for German
onehalfspacing, % Single line spacing, alternatives: onehalfspacing or doublespacing
%draft, % Uncomment to enable draft mode (no pictures, no links, overfull hboxes indicated)
%nolistspacing, % If the document is onehalfspacing or doublespacing, uncomment this to set spacing in lists to single
%liststotoc, % Uncomment to add the list of figures/tables/etc to the table of contents
%toctotoc, % Uncomment to add the main table of contents to the table of contents
%parskip, % Uncomment to add space between paragraphs
%nohyperref, % Uncomment to not load the hyperref package
headsepline, % Uncomment to get a line under the header
]{MastersDoctoralThesis} % The class file specifying the document structure

\usepackage[T1]{fontenc} % Output font encoding for international characters
\usepackage[utf8]{inputenc} % Required for inputting international characters


\usepackage{palatino} % Use the Palatino font by default
\usepackage[dvipsnames]{xcolor}

\usepackage[backend=bibtex,style=numeric,natbib=true]{biblatex} % Use the bibtex backend with the authoryear citation style (which resembles APA)

\addbibresource{references.bib} % The filename of the bibliography

\usepackage[autostyle=true]{csquotes} % Required to generate language-dependent quotes in the bibliography
\usepackage{algorithm}% http://ctan.org/pkg/algorithms
\usepackage{algpseudocode}% http://ctan.org/pkg/algorithmicx
\usepackage{amsmath}

%----------------------------------------------------------------------------------------
%	MARGIN SETTINGS
%----------------------------------------------------------------------------------------

\geometry{
	paper=a4paper, % Change to letterpaper for US letter
	inner=2.5cm, % Inner margin
	outer=3.8cm, % Outer margin
	bindingoffset=2cm, % Binding offset
	top=1.5cm, % Top margin
	bottom=1.5cm, % Bottom margin
	%showframe,% show how the type block is set on the page
}

%----------------------------------------------------------------------------------------
%	THESIS INFORMATION
%----------------------------------------------------------------------------------------
\thesistitle{Employing Hypergraphs for Efficient Coalition Formation with Application to the V2G Problem} % Your thesis title, this is used in the title and abstract, print it elsewhere with \ttitle
\supervisor{Georgios Chalkiadakis} % Your supervisor's name, this is used in the title page, print it elsewhere with \supname
\examiner{} % Your examiner's name, this is not currently used anywhere in the template, print it elsewhere with \examname
\degree{Electrical and Computer Engineer} % Your degree name, this is used in the title page and abstract, print it elsewhere with \degreename
\author{Filippos Christianos} % Your name, this is used in the title page and abstract, print it elsewhere with \authorname
\addresses{} % Your address, this is not currently used anywhere in the template, print it elsewhere with \addressname

\subject{} % Your subject area, this is not currently used anywhere in the template, print it elsewhere with \subjectname
\keywords{} % Keywords for your thesis, this is not currently used anywhere in the template, print it elsewhere with \keywordnames
\university{\href{http://www.tuc.gr}{Technical University of Crete}} % Your university's name and URL, this is used in the title page and abstract, print it elsewhere with \univname
\department{\href{http://www.ece.tuc.gr/}{School of Electrical and Computer Engineering}} % Your department's name and URL, this is used in the title page and abstract, print it elsewhere with \deptname
\group{} % Your research group's name and URL, this is used in the title page, print it elsewhere with \groupname
\faculty{\href{http://www.tuc.gr}{Technical University of Crete}} % Your faculty's name and URL, this is used in the title page and abstract, print it elsewhere with \facname

\hypersetup{pdftitle=\ttitle} % Set the PDF's title to your title
\hypersetup{pdfauthor=\authorname} % Set the PDF's author to your name
\hypersetup{pdfkeywords=\keywordnames} % Set the PDF's keywords to your keywords
\hypersetup{allcolors=Maroon}


\begin{document}

\frontmatter % Use roman page numbering style (i, ii, iii, iv...) for the pre-content pages

\pagestyle{plain} % Default to the plain heading style until the thesis style is called for the body content

%----------------------------------------------------------------------------------------
%	TITLE PAGE
%----------------------------------------------------------------------------------------

\begin{titlepage}
\begin{center}
	



{\scshape\LARGE \univname\par}\vspace{1.5cm} % University name
\textsc{\Large Thesis}\\[0.5cm] % Thesis type

\HRule \\[0.4cm] % Horizontal line
{\huge \bfseries \ttitle\par}\vspace{0.4cm} % Thesis title
\HRule \\[1.5cm] % Horizontal line
 
\emph{Author:}\\
{\authorname}\\ % Author name - remove the \href bracket to remove the link
\emph{Supervisor:} \\
{\supname} ({\em Associate Professor})\\ % Supervisor name - remove the \href bracket to remove the link  
 \vspace{30pt}
\large \textit{A thesis submitted in fulfillment of the requirements\\ for the diploma of \degreename}\\[0.3cm] % University requirement text
\textit{in the}\\[0.4cm]
\deptname\\\univname\\[2cm] % Research group name and department name
 
{\large \today}\\[4cm] % Date
%\includegraphics{Logo} % University/department logo - uncomment to place it
 
\vfill
\end{center}
\end{titlepage}

%----------------------------------------------------------------------------------------
%	QUOTATION PAGE
%----------------------------------------------------------------------------------------

%\vspace*{0.2\textheight}

%\noindent\enquote{\itshape Lorem ipsum dolor sit amet, consectetur adipiscing elit.}\bigbreak

%\hfill Lorem Ipsum
%----------------------------------------------------------------------------------------
%	ABSTRACT PAGE
%----------------------------------------------------------------------------------------


\begin{abstract}
\addchaptertocentry{\abstractname} % Add the abstract to the table of contents

This thesis proposes, for the first time in the literature, the use of {\em hypergraphs} for the efficient formation of effective agent coalitions.
We put forward several formation methods that build on existing hypergraph pruning, transversal, clustering and hybrid algorithms, 
and exploit the hypergraph structure to identify agents with desirable characteristics.
Our approach allows the near-instantaneous formation of high quality coalitions, adhering to multiple stated quality requirements.
Moreover, our methods are shown to scale to {\em dozens of thousands} of agents within fractions of a second; with one of them scaling to even {\em millions} of agents within seconds. We apply our approach to the problem of forming coalitions to provide {\em (electric) vehicle-to-grid (V2G)} services.
Ours is the first approach able to deal with {\em large-scale}, {\em real-time} coalition formation for the V2G problem, while taking {\em multiple criteria} into account for creating the electric vehicle coalitions. A sketch of these ideas appeared originally in a short paper in the 22nd European Conference on Artificial Intelligence (ECAI-2016). Afterwards, a full paper describing our work was published in the 14th European Conference on Multi-Agent Systems (EUMAS-2016).
\end{abstract}
\thesistitle{\textgreek{Χρήση υπεργράφων για την αποτελεσματική δημιουργία συνασπισμών με εφαρμογή στο πρόβλημα} V2G\textgreek{.}}
\author{\textgreek{Φίλιππος Χριστιανός}}

\university{\href{http://www.tuc.gr}{\greektext{Πολυτεχνείο Κρήτης}}} % Your university's name and URL, this is used in the title page and abstract, print it elsewhere with \univname
\department{\href{http://www.ece.tuc.gr/}{\greektext{Τμήμα Ηλεκτρολόγων Μηχανικών και Μηχανικών Υπολογιστών}}}

\degree{\greektext{Ηλεκτρολόγος Μηχανικός και Μηχανικός Υπολογιστών}} % Your degree name, this is used in the title page and abstract, print it elsewhere with \degreename
\faculty{\href{http://www.tuc.gr}{\greektext{Πολυτεχνείο Κρήτης}}} % Your faculty's name and URL, this is used in the title page and abstract, print it elsewhere with \facname


\providecaptionname{american,australian,british,canadian,english,newzealand,UKenglish,USenglish}{\byname}{\greektext{}}
\providecaptionname{american,australian,british,canadian,english,newzealand,UKenglish,USenglish}{\abstractname}{\greektext{Περίληψη}}

\begin{abstract}
	\selectlanguage{greek}
	\addchaptertocentry{\abstractname} % Add the abstract to the table of contents
	
	Αυτή η διπλωματική εισάγει, για πρώτη φορά στη βιβλιογραφία, την χρήση {\em υπεργράφων} για την ταχεία δημιουργία αποτελεσματικών συνασπισμών αυτόνομων πρακτόρων.
	Προτείνουμε ορισμένες μεθόδους σχηματισμού, που βασίζονται σε υπάρχοντες αλγορίθμους υπεργράφων, όπως οι \selectlanguage{english} pruning, transversal, clustering \textgreek{και} hybrid, \selectlanguage{greek} και εκμεταλλευόμαστε την δομή του υπεργράφου για να εντοπίσουμε πράκτορες με επιθυμητά χαρακτηριστικά.
	Η προσέγγισή μας επιτρέπει τον σχεδόν στιγμιαίο σχηματισμό συνασπισμών υψηλής ποιότητας, ικανοποιώντας πολλαπλές ποιοτικές απαιτήσεις.
	Επιπλέον, οι μέθοδοί μας κλιμακώνονται ώστε να δέχονται {\em δεκάδες χιλιάδες} πράκτορες ως είσοδο και να εμφανίζουν τα αποτελέσματα μέσα σε κλάσματα του δευτερολέπτου, με μια από αυτές να λειτουργεί με {\em εκατομμύρια} πράκτορες μέσα σε δευτερόλεπτα. Εφαρμόζουμε την προσέγγισή μας στο πρόβλημα της δημιουργίας συνασπισμών για την παροχή ρεύματος από {\em ηλεκτρικά οχήματα προς το ηλεκτρικό δίκτυο (το λεγόμενο πρόβλημα \selectlanguage{english}Vehicle-to-Grid\selectlanguage{greek}, ή\selectlanguage{english} V2G\selectlanguage{greek})}.
	Η προσέγγισή μας είναι η πρώτη που είναι σε θέση να ασχοληθεί με {\em μεγάλης κλίμακας}, και σε {\em πραγματικό χρόνο} σχηματισμό συνασπισμών για το πρόβλημα \selectlanguage{english}V2G\selectlanguage{greek}, λαμβάνοντας υπ'όψιν {\em πολλαπλά κριτήρια} για τη δημιουργία των συνασπισμών ηλεκτρικών οχημάτων. Ένα προσχέδιο των ιδεών αυτών εμφανίστηκε αρχικά σε μια σύντομη δημοσίευση στο 22ο \selectlanguage{english}{\em European Conference on Artificial Intelligence (ECAI-2016)}\selectlanguage{greek} και έπειτα σε μια πλήρη στο 14o \selectlanguage{english}{\em European Conference on Multi-Agent Systems (EUMAS-2016)}\selectlanguage{greek}.
	
\end{abstract}
\selectlanguage{english}

%----------------------------------------------------------------------------------------
%	ACKNOWLEDGEMENTS
%----------------------------------------------------------------------------------------
%
\begin{acknowledgements}
\addchaptertocentry{\acknowledgementname} % Add the acknowledgements to the table of contents

I would like to thank all the people who contributed in some way to the work described in this
thesis. First and foremost, I would like to express my gratitude to my supervisor Georgios Chalkiadakis for his patience, motivation and knowledge. His guidance helped me immensely during the research and writing of this thesis. 

Besides my advisor, I would like to thank Stefanos, Dimitris, Tonia and Panagiotis for being great friends and offering me invaluable support throughout these years. I would like to especially thank Dafne for all her love, support and helpful advice.

To my family, thank you for encouraging me in all
of my pursuits and inspiring me to follow my dreams. I am especially grateful to my parents, Rania and Vassilis, who supported me emotionally and financially and always wanted the best for me. This thesis is dedicated to them.
\end{acknowledgements}

%----------------------------------------------------------------------------------------
%	LIST OF CONTENTS/FIGURES/TABLES PAGES
%----------------------------------------------------------------------------------------

\tableofcontents % Prints the main table of contents

\listoffigures % Prints the list of figures

\listoftables % Prints the list of tables

%----------------------------------------------------------------------------------------
%	ABBREVIATIONS
%----------------------------------------------------------------------------------------

%\begin{abbreviations}{ll} % Include a list of abbreviations (a table of two columns)

%\textbf{CF} & \textbf{C}oalition \textbf{F}ormation\\
%\textbf{CS} & \textbf{C}oalition \textbf{S}tructure\\
%\textbf{CSF} & \textbf{C}oalition \textbf{S}tructure \textbf{G}eneration\\
%\textbf{EV} & \textbf{E}lectric \textbf{V}ehicle\\
%\textbf{G2V} & \textbf{G}rid to \textbf{V}ehicle\\
%\textbf{V2G} & \textbf{V}ehicle to \textbf{G}rid\\
%\textbf{SoC} & \textbf{S}tate of \textbf{C}harge\\
%\textbf{MAS} & \textbf{M}ulti\textbf{A}gent \textbf{S}ystems\\


%\end{abbreviations}


%----------------------------------------------------------------------------------------
%	THESIS CONTENT - CHAPTERS
%----------------------------------------------------------------------------------------

\mainmatter % Begin numeric (1,2,3...) page numbering

\pagestyle{thesis} % Return the page headers back to the "thesis" style

% Include the chapters of the thesis as separate files from the Chapters folder
% Uncomment the lines as you write the chapters

\include{Chapters/Chapter1_intro}
% Chapter 1
\chapter{Background}
\label{Chapter2_bg}
This chapter presents some background for the research presented in this thesis. Specifically, section~\ref{Chapter2_evs} starts with a definition and overview of the Smart Grid and connects it to electric vehicles. In section~\ref{Chapter3_cf} we discuss coalition formation and finally, in section~\ref{Chapter4_hg} we present background information on hypergraphs.

\section{Electric Vehicles in the Smart Grid} % Main chapter title

\label{Chapter2_evs} % For referencing the chapter elsewhere, use \ref{Chapter1} 
%\label{sec:introduction}
%----------------------------------------------------------------------------------------

%----------------------------------------------------------------------------------------

The current electricity Grid, is the network that delivers power to consumers. It uses large central power stations that distribute the energy through high capacity power lines to both industrial and domestic areas. Historically the Grid handled peak hours poorly, with blackouts and power cuts being common~\cite{gorowitz2000general}. Only more recently and after establishing patterns in electricity demands, could the daily peaks be met, using part-time generators (usually the expensive gas turbines). The current structure of the Grid is a product of an evolutionary process that lasted decades, connected to growing needs of consumers. Thus, the current infrastructure was not planned as a whole, but rather extended several times, creating weak links and containing outdated designs. One of the most important issues, is the centralized nature of the Grid. It is built around large power plants producing all the energy required by the consumers. With the growth of smaller, usually renewable, energy producers, the Grid by necessity has to move to a less centralized and more interactive structure.

The Smart Grid, therefore, is a modernized electricity Grid that collects and uses information to improve efficiency, reliability, economics and sustainability of the electrical Grid. It is also planned to be more decentralized making efficient use of small scale producers and prosumers.~\footnote{Prosumer is a small scale electricity consumer that might also produce energy\cite{lampropoulos2010methodology}.} Thus, the Smart Grid can potentially become much more reliable than its "classic" counterpart by eliminating single points of failure. Another important characteristic of the Smart Grid is that it is much more interactive. Everyone connected, communicates and coordinates with the Grid, rendering the consumption and production more balanced. Connected consumers also implement smart technologies that drive their own consumption down. With techniques like the ones mentioned above, the network's energy load is balanced and thus the distribution is more efficient - eliminating, if possible, high-cost energy producers like gas turbines. To understand the efficiency of the Smart Grid, experiments have, for example, shown that just a small scale coordination of a few battery-equipped houses can lower everyone's electric bill~\cite{vytelingum2010agent}.
%\section{Electric Vehicles}

Electric vehicles (EVs) are a promising new concept for the automotive industry. EVs use energy stored in a battery and electric motors to generate propulsion. Electricity offers many advantages against petrol-powered vehicles. Specifically, EVs are cost effective, require less maintenance, and have no direct emissions since they run in electricity powered engines. In their current state, the batteries of electric vehicles (which rapidly become even more efficient and cost effective~\cite{nykvist2015rapidly}) are capable of at least 300km~\cite{globalev2016}\cite{young2013electric} of range.~\footnote{The range of an EV is defined as the driving range using power only from its battery pack during a single charge.} To achieve this range, the batteries have a large capacity usually in the 60kWh-100kWh range. To put this into perspective, batteries as low as 4kWh can have a significant impact on the energy footprint of a typical household~\cite{vytelingum2010agent}.

Another important factor for the battery is the discharge rate. A vehicle requires a large amount of energy during acceleration. For example, accelerating a typical vehicle to 100km/h in 10 seconds can require up to 65kW of power~\cite{young2013electric}. Since EV batteries are actually designed for discharging at these rates for typical driving, we can safely assume that we can use this discharge rate for other uses.

Charging the batteries is another characteristic that must be accounted for. Currently, charging the battery takes a few hours depending on the battery's state of charge (Soc). Nevertheless for an everyday use scenario, a battery can be expected to charge (using fast charging) to a reasonable amount in half an hour~\cite{young2013electric}. In this thesis we will not examine how EVs charge, or how we can regulate its charging.

Due to the previously mentioned growth of EVs and their energy capacity we can safely assume that they will play a significant role in the future of the electricity Grid~\cite{ramchurn2012putting}. As a result, two categories of problems arise. First, the issue of how we can successfully provide the energy those vehicles need, and charge them without overloading the Grid. The second is how energy stored in EVs can be used to balance out peaks in consumption or even serve as backup power. Those categories are called Grid to Vehicle (G2V) and Vehicle to Grid (V2G)~\cite{loisel2014large}~\cite{kempton2005vehicle} respectively.

G2V is better explained by noticing that, due to the common working hours, large numbers of EV owners might return home and charge their vehicles, at about the same time. Since EVs can draw a huge amount of power the Grid will overload due to huge spikes on consumption. Nevertheless the charging could have been coordinated and EVs charged during the night, without causing spikes. Finding an optimal way to charge EVs this way though is quite complicated, since possibly millions of batteries have to eventually be charged. Several attempts have been made to tackle the problem~\cite{gan2013optimal}\cite{valogianni2014effective} but are not usually scalable to large numbers of EVs.

V2G, a problem related to our work here, in contrast to G2V, is the question of how EVs can supply power (usually stored in the vehicle's battery) to the Smart Grid during power peaks. This can lower or even eliminate the need of expensive back up generators. Since the batteries can charge when power is cheap (e.g. at night) and return the power when its more expensive, this raises an opportunity for profit for EV owners. Nevertheless this does not come without several issues to be addresses. Specifically, a single EV must know if its owner will need the energy that will be sent to the Grid and not participate in an exchange if there's a chance the owner needs to use the vehicle. In addition, EVs can be used by owners without any previous notice and might be unplugged from the electricity Grid at any moment. This raises the issue of reliability: how certain we are that a vehicle that promises to deliver power during a timeslot, can actually keep its promise. Finally, while the batteries can store and provide a respectable amount of energy, the needs of the Grid are proportionally much greater. Thus EVs must be able to cooperate to provide sufficient and reliable services. Several aspects of V2G have been researched. We will mention several such attempts in Chapter~\ref{Chapter5_related}.


%Electric vehicles (EVs) are a promising new concept for the automotive industry. EVs use energy stored in a battery and electric motors to generate propulsion. Electricity offers many advantages %against petrol-powered vehicles. Specifically, EVs are cost effective and require less maintenance, and thus have no emissions since they run in electricity powered engines. The growing popularity of EVs %gives rise to the so-called G2V and V2G problems. G2V describes a system where EVs connect and draw power from the Grid without overloading it\cite{valogianni2014effective}. V2G is the problem of EVs %communicating with the Grid in order to either lower their power demands or return power back to the network when there is a peak in the request for power. This helps the Grid to maintain a balanced power %load\cite{ramchurn2012putting}. This is the problem we will be dealing with in this paper.

%	An important issue in the V2G problem is that there are possibly millions of EVs which communicate and connect to the Grid. The vast number of vehicles means that we must create the most appropriate %groups to cover the needs of the Grid at any given time. Algorithms that scale well and give results almost instantly are necessary. 

%	In order to tackle the V2G problem, we resort to coalition formation. Specifically, we propose the formation of coalitions using hypergraphs. By doing so, we can efficiently locate reliable agents and form %effective EV cooperatives to provide sufficient energy and stability. 

%	Such attempts use mostly machine learning or attempting to form the optimal coalition\cite{deORamos2014}\cite{valogianni2014effective}. This had the drawback that it did not scale to more than a few %hundred agents \cite{kamboj2010exploring} \cite{deORamos2014}. Besides, the approaches that have been used do not deal with multi-criteria optimization. This is important, however because in reality %coalitions have to be formed according to several criteria such as capacity and discharge rate. In our attempt, we will try to form coalitions by selecting vehicles from a huge pool of individual EVs using 
% multiple criteria for our selections. 


%	The Grid should be able to advertise the amount of power it requires by both asking for a required capacity and a maximum discharge rate. What we are trying to do is fulfill the required capacity and %discharge rate with the minimum amount of vehicles and by keeping our coalition reliable. We are not searching for an optimal coalition but rather for one that can be generated quickly and reliably. We do %this by organizing our electric vehicles inside a hypergraph. Current research and solutions on the V2G problem do not scale well. It should be noted that it is also the first attempt to use hypergraphs for %coalition generation. Hypergraphs are well studied, and powerful algorithms do exist for traversing and exploring them. 

%	In a few words, we start with a huge pool of EVs. We know their power capacity, discharge rate and if they are committed to connect to the Grid. We also know their reliability. The Grid advertises the demand of a coalition with a specific capacity and discharge rate. We form a coalition that fulfills the power requirements and has a high reliability while also being small in size. 
%	Now in order to build coalitions for the V2G we need to combine the capabilities of EVs. This naturally gives rise to a multi-criterion selection problem for choosing the members of a coalition. In order to tackle this problem, we propose a novel, principled approach in order to form coalitions that have specific characteristics. For this we employ the use of hypergraphs and research that has been done on them \cite{zhou2006learning} \cite{kavvadias2005efficient}

%	In general the related work\cite{kamboj2010exploring}\cite{kamboj2011deploying}\cite{deORamos2014}\cite{valogianni2014effective} focuses in single-criteria coalition formation and in near-optimal %solutions that require great processing power and scale poorly. 

%  ***** AYTO as to exoume ypopsin gia pi8anes erwthseis ****
%First of all, while multiple criteria can usually be expressed as a single one with the help of a utility function, we find that multi-criteria is a more natural way to express the agent's attributes. As such we do not use directly utility functions.

%	To continue while finding the optimal coalition is useful in most cases it can't work in real world situations where there could be millions of EVs, and the requirements could be updated every few seconds. Therefore, we sacrifice the ability to find the optimal solution so that we can process thousands of EV's in a few seconds. 
% Chapter 1

\section{Coalition Formation} % Main chapter title

\label{Chapter3_cf} % For referencing the chapter elsewhere, use \ref{Chapter1} 
%\label{sec:introduction}
%----------------------------------------------------------------------------------------

%----------------------------------------------------------------------------------------
Coalition Formation deals with how agents can form one or more groups, called coalitions, that can tackle a common problem. CF theory analyzes several of its aspects, that range from creating such coalitions to fairly dividing rewards to the members of a coalition.  
Individual agents usually have different degrees of efficiency. Thus, we must form groups of agents with characteristics that compliment each other and exploit their individual strengths\cite{shehory1998methods}.
As discussed in~\cite{sandholm1999coalition}, coalition formation has three activities. {\em Coalition structure generation} (CSG), is the first of these activities, namely the partitioning of the set of agents into mutually disjoint coalitions (or groups), in a way that the resulting coalitions maximize the sum of the rewards of all agents (known as social welfare)~\cite{rahwan2009anytime}. Next is {\em the optimizations problem} of each coalition, that tries to maximize the rewards from outside the coalition and optimize the allocation of resources and tasks between agents of the respective coalition. The last activity of CF is the {\em division of the rewards} among agents. This must be done in such a way that the rewards are fair, and no agent can be motivated to leave his coalition.

Finding the optimal coalition structure is generally computationally expensive, especially in a large set of agents, and the computational requirements grow exponentially. As such, finding in a reasonable time a CS that is within a bound of the optimal one, is also hard problem. There are several attempts to solve this problem~\cite{sandholm1999coalition}~\cite{rahwan2009anytime}.

Nevertheless, in this thesis we will not attempt to solve the CSG problem. Instead we will focus on finding a simple coalition that is able to perform a specific task. Such a coalition will be created by selecting agents from an extensive set, in a way that the result can efficiently handle the appointed task. In addition each agent will have several attributes that contribute in different ways to the completion of the goal. We will not be using a utility function (that ultimately combines the attributes, thus losing accuracy within its particular dimensions). 

By contrast, in essence we will be tackling what we call {\em multi-criteria} coalition formation, that is, the problem of forming coalitions that can accomplish a task that requires meeting a range of task-related goals: for instance, offering a minimal value of charging capacity, a minimal value of charging rate and so on. Due to the nature of this problem, the results cannot be easily evaluated. There is possibly a great number of possible coalitions with similar capacity to handle the task. In addition, since the agent set is magnitudes larger that what usual optimal CF algorithms can handle we cannot find how close to optimality our solution is. Thus we will focus on creating coalitions that can complete the task efficiently and can be generated in a minimal amount of time.

This problem can be quite natural in todays world. There are several real world examples where efficiency is sacrificed for performance. In this thesis, we will present a way to form an EV coalition in just a few seconds, able to fulfill the energy requirements of the Smart Grid.

%----

%Electric vehicles (EVs) are a promising new concept for the automotive industry. EVs use energy stored in a battery and electric motors to generate propulsion. Electricity offers many advantages %against petrol-powered vehicles. Specifically, EVs are cost effective and require less maintenance, and thus have no emissions since they run in electricity powered engines. The growing popularity of EVs %gives rise to the so-called G2V and V2G problems. G2V describes a system where EVs connect and draw power from the Grid without overloading it\cite{valogianni2014effective}. V2G is the problem of EVs %communicating with the Grid in order to either lower their power demands or return power back to the network when there is a peak in the request for power. This helps the Grid to maintain a balanced power %load\cite{ramchurn2012putting}. This is the problem we will be dealing with in this paper.

%	An important issue in the V2G problem is that there are possibly millions of EVs which communicate and connect to the Grid. The vast number of vehicles means that we must create the most appropriate %groups to cover the needs of the Grid at any given time. Algorithms that scale well and give results almost instantly are necessary. 

%	In order to tackle the V2G problem, we resort to coalition formation. Specifically, we propose the formation of coalitions using hypergraphs. By doing so, we can efficiently locate reliable agents and form %effective EV cooperatives to provide sufficient energy and stability. 

%	Such attempts use mostly machine learning or attempting to form the optimal coalition\cite{deORamos2014}\cite{valogianni2014effective}. This had the drawback that it did not scale to more than a few %hundred agents \cite{kamboj2010exploring} \cite{deORamos2014}. Besides, the approaches that have been used do not deal with multi-criteria optimization. This is important, however because in reality %coalitions have to be formed according to several criteria such as capacity and discharge rate. In our attempt, we will try to form coalitions by selecting vehicles from a huge pool of individual EVs using 
% multiple criteria for our selections. 


%	The Grid should be able to advertise the amount of power it requires by both asking for a required capacity and a maximum discharge rate. What we are trying to do is fulfill the required capacity and %discharge rate with the minimum amount of vehicles and by keeping our coalition reliable. We are not searching for an optimal coalition but rather for one that can be generated quickly and reliably. We do %this by organizing our electric vehicles inside a hypergraph. Current research and solutions on the V2G problem do not scale well. It should be noted that it is also the first attempt to use hypergraphs for %coalition generation. Hypergraphs are well studied, and powerful algorithms do exist for traversing and exploring them. 

%	In a few words, we start with a huge pool of EVs. We know their power capacity, discharge rate and if they are committed to connect to the Grid. We also know their reliability. The Grid advertises the demand of a coalition with a specific capacity and discharge rate. We form a coalition that fulfills the power requirements and has a high reliability while also being small in size. 
%	Now in order to build coalitions for the V2G we need to combine the capabilities of EVs. This naturally gives rise to a multi-criterion selection problem for choosing the members of a coalition. In order to tackle this problem, we propose a novel, principled approach in order to form coalitions that have specific characteristics. For this we employ the use of hypergraphs and research that has been done on them \cite{zhou2006learning} \cite{kavvadias2005efficient}

%	In general the related work\cite{kamboj2010exploring}\cite{kamboj2011deploying}\cite{deORamos2014}\cite{valogianni2014effective} focuses in single-criteria coalition formation and in near-optimal %solutions that require great processing power and scale poorly. 

%  ***** AYTO as to exoume ypopsin gia pi8anes erwthseis ****
%First of all, while multiple criteria can usually be expressed as a single one with the help of a utility function, we find that multi-criteria is a more natural way to express the agent's attributes. As such we do not use directly utility functions.

%	To continue while finding the optimal coalition is useful in most cases it can't work in real world situations where there could be millions of EVs, and the requirements could be updated every few seconds. Therefore, we sacrifice the ability to find the optimal solution so that we can process thousands of EV's in a few seconds.
% Chapter 1

\section{Hypergraphs} % Main chapter title

\label{Chapter4_hg} % For referencing the chapter elsewhere, use \ref{Chapter1} 
%\label{sec:introduction}
%----------------------------------------------------------------------------------------

%----------------------------------------------------------------------------------------

The Hypergraph, is a generalization of a graph. In contrast to a simple graph, hypergraph's edges can connect multiple vertices. We formally define a hypergraph as $H = (V, E)$ where $V$ is a set of unique {\em vertices} or {\em nodes} and $E$ is the set of {\em edges} or {\em hyperedges}. 
\begin{figure}
	\includegraphics[scale=1]{simple_hg.png}
	\caption{A Simple Hypergraph}
	\label{simplehg}
\end{figure}

A simple hypergraph can be seen in figure~\ref{simplehg}. In this example, the vertices of the hypergraph are $V = \{u_1, u_2, u_3, u_4\}$ and the edges $E = \{e_1, e_2\}$. Edge $e_1$ contains vertices $u_1, u_2$, while $e_2$ contains $u_2, u_3, u_4$.

The difference of graphs and hypergraphs is outlined in the figure of the hyperedge $e_2$~\ref{hyperedge}. As shown, a hyperedge can connect multiple vertices, in contrast to simple graphs where an edge always connects two nodes.

\begin{figure}
	\includegraphics[scale=1]{simple_hyperedge.png}
	\caption{A Single Hyperedge}
	\label{hyperedge}
\end{figure}

Hypergraphs are well studied, and thus there are many definitions in literature which can help us use them as a data structure. First, a {\em transversal}, or {\em hitting set}, of a hypergraph is a set of nodes $T \subset V$ such that T intersects with any edge $E$ of the hypergraph. A {\em hitting set} that does not contain any other hitting set is called {\em minimal}. This is better illustrated in Fig.~\ref{transversalexample}. As shown, the hypergraph has three {\em minimal transversals}: $T_1 = \{u_1, u_4\}, T_2 = \{u_1, u_3\}$ and $T_3 = \{u_2\}$. Any other transversal such as $T_4 = \{u_2, u_4\}$ is not minimal since $T_3 \subset T_4$ (it contains at least one other transversal). Finally, the {\em set of minimal transversals} is called the {\em dual} of a hypergraph or {\em minimal set-cover} and denoted as $H^d$. 

We expect this problem to be of high complexity since it is essentially the set-cover problem extended to hypergraphs. The search version of the set-cover problem in graphs is NP-hard. However, computing the dual hypergrah is a problem widely studied, and hence, many algorithms offer efficient solutions. To begin, Berge's algorithm~\cite{berge1989hypergraphs}, while slower than the rest, is foundation of many algorithms which are merely an improvement on it. Specifically, the algorithms described here, are divided in two types: improvements over Berge and hill-climbing algorithms. The Berge algorithm updates the set of minimal transversals by iteratively adding hyperedges.

Dong and Li~\cite{dong2005mining}, for instance, is a Berge-based algorithm that improves upon it. This method decreases the search space by avoiding to generate several non-minimal transversals. Bailey {\em et al.}~\cite{bailey2003fast} algorithm, starts with a limited vertex set and update both hyperedges and hitting sets by constantly adding new vertices. Kavvadias {\em et al.}
~\cite{kavvadias2005efficient} propose a memory-bound depth-first approach which generates a constant stream of minimal hitting sets. Nevertheless this method does not come with a time complexity bound. Khachiyan {\em et al.}~\cite{boros2003efficient} provide a {\em quasi-polynomial} algorithm for enumerating all minimal transversals. In a sense of time complexity this is the fastest of the algorithms presented here, but in practice it falls behind in several test-cases. Extending on that, another publication~\cite{khachiyan2005new} finds a multi-threaded solution with excellent time complexity assuming multiple cores. Finally~\cite{hebert2007data}, a solution not based on Berge, offers a hill-climbing algorithm that adds vertices in increasing order and checks if they satisfy a minimal transversal condition.
\begin{figure}
	\includegraphics[scale=1]{transversal_example.png}
	\caption{Color-coded Transversals}
	\label{transversalexample}
\end{figure}


Another well-studied hypergraph-related problem is hypergraphs clustering. This is done by regarding the edges as node attributes and using several techniques to cluster nodes with similar attributes together. For instance~\cite{zhou2006learning}, proposes powerful methods of spectral clustering on hypergraphs and algorithms for classification and embedding. The methods proposed had a significant advantage when used in hypergraphs over simple graphs, since they managed to store complex relationships among objects on the hyperedges. A game theoretic approach to hypergraph clustering is found in~\cite{bulo2009game}. Specifically, there the cluster is treated as the game-theoretic concept of equilibrium, and the problem of partitioning the hypergraph to clusters as non-cooperative multiplayer game. This has several advantages over classical approaches, e.g. the final number of clusters is not needed beforehand. Finally, Leordeanu and Sminchisescu~\cite{leordeanu2012efficient} propose an efficient clustering method that updates which vertices correspond to which clusters in parallel through an iterative procedure. This manages to reduce computing requirements significantly.



In addition, there are also many ways for matrices to represent a hypergraph or specific attributes of it. For instance the incidence matrix, weight matrix and adjacency matrix are all easily defined, as we show in Section~\ref{sec:Clustering}.

Hypergraphs thus, are a powerful and well-defined way to store information. For example we can easily store (better explained in Chapter~\ref{Chapter6_approach}) EVs in hyperedges that represent a specific quality. Edges then store both attributes of EVs and, possibly, complex relations between them. This comes with advantages like the instant selection of interesting parts of the graph or fast set operations like {\em intersection} or {\em union}.

Later, in Chapter~\ref{Chapter6_approach} we explain how these advantages and the existing literature is exploited for efficient coalition formation.



%----

%Electric vehicles (EVs) are a promising new concept for the automotive industry. EVs use energy stored in a battery and electric motors to generate propulsion. Electricity offers many advantages %against petrol-powered vehicles. Specifically, EVs are cost effective and require less maintenance, and thus have no emissions since they run in electricity powered engines. The growing popularity of EVs %gives rise to the so-called G2V and V2G problems. G2V describes a system where EVs connect and draw power from the Grid without overloading it\cite{valogianni2014effective}. V2G is the problem of EVs %communicating with the Grid in order to either lower their power demands or return power back to the network when there is a peak in the request for power. This helps the Grid to maintain a balanced power %load\cite{ramchurn2012putting}. This is the problem we will be dealing with in this paper.

%	An important issue in the V2G problem is that there are possibly millions of EVs which communicate and connect to the Grid. The vast number of vehicles means that we must create the most appropriate %groups to cover the needs of the Grid at any given time. Algorithms that scale well and give results almost instantly are necessary. 

%	In order to tackle the V2G problem, we resort to coalition formation. Specifically, we propose the formation of coalitions using hypergraphs. By doing so, we can efficiently locate reliable agents and form %effective EV cooperatives to provide sufficient energy and stability. 

%	Such attempts use mostly machine learning or attempting to form the optimal coalition\cite{deORamos2014}\cite{valogianni2014effective}. This had the drawback that it did not scale to more than a few %hundred agents \cite{kamboj2010exploring} \cite{deORamos2014}. Besides, the approaches that have been used do not deal with multi-criteria optimization. This is important, however because in reality %coalitions have to be formed according to several criteria such as capacity and discharge rate. In our attempt, we will try to form coalitions by selecting vehicles from a huge pool of individual EVs using 
% multiple criteria for our selections. 


%	The Grid should be able to advertise the amount of power it requires by both asking for a required capacity and a maximum discharge rate. What we are trying to do is fulfill the required capacity and %discharge rate with the minimum amount of vehicles and by keeping our coalition reliable. We are not searching for an optimal coalition but rather for one that can be generated quickly and reliably. We do %this by organizing our electric vehicles inside a hypergraph. Current research and solutions on the V2G problem do not scale well. It should be noted that it is also the first attempt to use hypergraphs for %coalition generation. Hypergraphs are well studied, and powerful algorithms do exist for traversing and exploring them. 

%	In a few words, we start with a huge pool of EVs. We know their power capacity, discharge rate and if they are committed to connect to the Grid. We also know their reliability. The Grid advertises the demand of a coalition with a specific capacity and discharge rate. We form a coalition that fulfills the power requirements and has a high reliability while also being small in size. 
%	Now in order to build coalitions for the V2G we need to combine the capabilities of EVs. This naturally gives rise to a multi-criterion selection problem for choosing the members of a coalition. In order to tackle this problem, we propose a novel, principled approach in order to form coalitions that have specific characteristics. For this we employ the use of hypergraphs and research that has been done on them \cite{zhou2006learning} \cite{kavvadias2005efficient}

%	In general the related work\cite{kamboj2010exploring}\cite{kamboj2011deploying}\cite{deORamos2014}\cite{valogianni2014effective} focuses in single-criteria coalition formation and in near-optimal %solutions that require great processing power and scale poorly. 

%  ***** AYTO as to exoume ypopsin gia pi8anes erwthseis ****
%First of all, while multiple criteria can usually be expressed as a single one with the help of a utility function, we find that multi-criteria is a more natural way to express the agent's attributes. As such we do not use directly utility functions.

%	To continue while finding the optimal coalition is useful in most cases it can't work in real world situations where there could be millions of EVs, and the requirements could be updated every few seconds. Therefore, we sacrifice the ability to find the optimal solution so that we can process thousands of EV's in a few seconds. 
% Chapter 1

\chapter{Related Work} % Main chapter title

\label{Chapter5_related} % For referencing the chapter elsewhere, use \ref{Chapter1} 

%----------------------------------------------------------------------------------------

% Define some commands to keep the formatting separated from the content 


%----------------------------------------------------------------------------------------

%\label{sec:related}
Here we review related work mainly on the V2G and the SCG problems, and highlight its differences to our approach in this thesis.
To begin, in their pioneer work, Valogianni {\em et al.}~\cite{valogianni2014effective} propose an {\em adaptive smart charging algorithm} that adjusts {\em the power drawn} from the electrical Grid for charging EVs, based on each EV owner's utility from charging. backbone of the approach is based on a {\em reinforcement learning} for capturing agent needs and behavior. It also utilizes an optimization module that schedules the charging of each EV in order to maximize its utility, subject to network constraints. Though effective, this work fails to focus on the problem of feeding the network with power drawn from EVs in a coordinated fashion. As such EV coalitions and their potential are not being considered in this work.

Contrary to the aforementioned line of research, the work presented in~\cite{kamboj2010exploring} considers an attempt to exploit EV coalition formation in energy exchange. In particular in~\cite{kamboj2010exploring} EV coalitions are utilized in selling power in the regulation market. In more detail, EV coalitions provide the following service to the Grid every few seconds: they, either, (i) {\em scale down} their power draw (or discharge); or they (ii)  {\em scale it up}, and request more power from the Smart electrical Grid.
Despite the effectiveness of this approach, there are considerable limitations with respect to its practical application that renders its usability in real setting scenario, questionable. In more detail there is a need for a complicated and resource-consuming EV selection process by an aggregator agent. Moreover, in this context, and to limit the respective complexity, the simulations involved in this work considered a limited pool of three hundred vehicles only.

That said, the potential of coalition formation is not only exploited in the narrow context of EVs. In more detail, coalition formation has long been investigated to provide regulation services to the Smart Grid and, in recent years several works in this direction emerged. For instance the work in~\cite{vinyals2012stable} adapts a game-theoretic perspective on the formation of coalitions in the Smart Grid. In this context, it considers the optimal coalition structure generation problem (CSG). To this end, it utilizes an approach of forming {\em virtual energy consumer} coalitions. In the context of these coalitions, it manages to flatten the energy demand. This, in turn, enhances the negotiational ability with the Grid, enabling better prices in what could be a G2V arrangement. In more detail, the solution of the CSG, provides the best VEC for every consumer on the market; and guarantees a core-stable payoff distribution outcome. Nevertheless, the computational complexity of this approach renders it impractical in real settings. In particular, this work has been shown to perform adequately on social graphs of limited size (with only a handful of agents). Notably, against this background, our approach manages to produce high quality solutions in milliseconds, and scales to the number of millions (as further discussed in Chapter~\ref{Chapter7_results}).

Now, two recent papers which study {\em cooperative games} defined {\em over graphs} that impose constraints on the formation of the coalitions, are~\cite{chalkiadakis2016characteristic} and~\cite{chalkiadakis2012coalitional}. Specifically, they assume that the environment possesses some structure that forbids the creation of individual coalitions, due to limited resources and existing physical or even legal barriers. This is captured by an undirected graph providing a path connecting any two agents that can belong to the same coalition.
%Communication, transportation, or sensor networks provide natural settings for cooperative games over graphs.
Both of these papers, however, do not employ hypergraphs in any way. Hypergraphs have in fact been used for modelling agent interactions in cooperative game settings, where agents can simultaneously belong to multiple coalitions~\cite{jun2009hypergraph} \cite{zick2012overlapping}. Now, several papers~\cite{chalkiadakis2016characteristic}~\cite{chalkiadakis2012coalitional}~\cite{jun2009hypergraph}~\cite{zick2012overlapping} focus on studying the theoretical problem of achieving {\em coalitional stability} via appropriately distributing the payoff among the agents. This is done rather than providing algorithms for large-scale coalition formation in real-world settings, as we do in this work. 

By contrast, two papers that study the generation of optimal coalition structures while focusing on stability are~\cite{bistaffa2014anytime}~\cite{voice2012coalition}. They focus on the use of synergy graphs. Those graphs connect agents with edges that represent a vital synergistic link, such as communication, trust or physical constrains. They propose efficient ways to generate all possible coalitions and find the optimal coalition structure.  Although these approaches scale to thousands of agents they are limited in terms of scalability compared to our approach which scales to millions of users. Furthermore, their approach fails to tackle multiple formation criteria.

A paper that is more related to our work here, in the sense that it exploits constraints among vehicles for coalition formation, is the work of Ramos {\em et al.}~\cite{deORamos2014}. In this context, they propose the dynamic formation of coalitions among EVs so that they can function as {\em virtual power plants} that sell power to the Grid as an aggregate. The method relies heavily on a inter-agent negotiations protocol. However, that work also attempts to tackle the optimal CSG problem and hence suffers from high complexity and scalability issues. As such, although it is empirically shown to produce solutions that are close to optimal (98\%), this is only when tested in scenarios with a few dozens of agents. In addition, the work in~\cite{deORamos2014} considers only a single criterion for the formation of a coalition---namely, the {\em physical distance} among the EVs. The physical distance, however, is not a very natural criterion; and, in any case, it is imperative that a multitude of criteria is taken into account---such as capacity, discharge power, and perceived reliability (see, e.g.,~\cite{kamboj2011deploying}). Our approach, in contrast, is able to take into account any number of natural criteria to form EV coalitions.
% Chapter 1

\chapter{Our Approach} % Main chapter title

\label{Chapter6_approach} % For referencing the chapter elsewhere, use \ref{Chapter1} 

%----------------------------------------------------------------------------------------


%----------------------------------------------------------------------------------------



%\section{}\label{sec:approach}



\begin{figure}
	\centering
	\includegraphics[scale=1]{hypergraph.png}
	\caption{Storing EVs in a hypergraph\label{fig:hypergraph}}
\end{figure}
In order to develop multi-criteria coalition formation algorithms that generate coalitions efficiently, we employ the concept of {\em a hypergraph}.
A hypergraph $H = (V, E)$ is a generalization of a graph, where each {\em hyperedge} $e \in E$ can contain any number of {\em vertices (or nodes)} in the set $V$.

Vertices in $H$ correspond to agents; while we view a hyperedge as corresponding to some particular {\em attribute} or {\em characteristic} possessed by the agents in the hyperedge.
In the V2G setting, the agents correspond to EVs (i.e., an EV is represented by a node in our hypergraph); while the hyperedges correspond to vehicle characteristics. More specifically, a hyperedge corresponds to a ``quality level'' of some EV attribute, as we explain below.

In order to represent the different {\em quality} of the various hyperedges, and utilize it in our algorithms, we mark each hyperedge with a weight.\footnote{In our implementation, the weight of the edges, according to the quality of each attribute(capacity, reliability and discharge), are as follows: \{\textit{extremely-high: 8, very-high: 7, high: 6, medium-high: 5, medium-low: 4, low: 3,very-low: 2, extremely-low:1}\}. Thus we have 24 edges + 1 containing commitment of EVs.} These weights define the \textit{degree} of a node: {\em The degree $deg(u)$ 
	of a node 
	$u$ 
	is the sum of the weights of its edges}. Intuitively, {\em a high degree node is a high quality one}. This fact is exploited in our algorithms below.
A hyperedge (of a given quality) will be also called a {\em category}. The (quality of the) categories to which an EV belongs will be influencing the decisions of our {\em hypergraph pruning} algorithm, which we describe in Section~\ref{sec:pruning} below. A node that belongs to a hyperedge characterizing the quality of a given agent attribute, cannot belong to some other hyperedge characterizing the quality of the same attribute.
\begin{figure}
	\centering
	\centering
	\includegraphics[scale=1]{hypergraph_prune.png}
	\caption{Pruning the hypergraph\label{fig:hypergraph_prune}}
	
\end{figure}

To illustrate the use of hypergraphs in our setting, consider for example the hypergraph of Fig.~\ref{fig:hypergraph}, which contains the hyperedges $e_{1...6}$ and vertices $u_{1...7}$. It is clear in this example that vertices may belong to multiple hyperedges: the hyperedge $e_1$ contains the vertices $u_{3,4,6,7}$, while the vertex $u_1$ belongs in the hyperedges $e_2, e_5, e_4$. Vertices in Fig.~\ref{fig:hypergraph} correspond to EVs; while the hyperedges correspond to the ``quality'' of the following EV attributes: \textit{capacity}, {\em discharge rate} and \textit{observed reliability}. The meaning of these attributes is intuitively straightforward, but will be nevertheless explained in Section~\ref{subsec:criteria} below. Each attribute is related to at least one hyperedge in the hypergraph. For instance, in Fig.~\ref{fig:hypergraph}, the {\em capacity} attribute is represented by three hyperedges in the hypergraph: {\em low-capacity}, {\em medium-capacity}, and {\em high-capacity}. As noted above, no node can belong in more than one capacity-related hyperedges. In our figure, 
\begin{itemize}
	\item the hyperedge $e1$ represents the nodes which have high capacity;
	\item the hyperedge $e2$ contains nodes that have low capacity;
	\item $e3$ and $e4$ include the vehicles with high and low discharge rate, respectively;
	\item finally, $e5$ contains nodes that are expected to the {\em highly reliable}.
\end{itemize}
For example, node $u1$ is a {\em low-capacity}, {\em low-discharge} but {\em highly reliable} vehicle, while node $u3$ is a {\em high-capacity}, {\em low-discharge} and {\em highly reliable} one.

%       A simplified example of how we store attributes of a hypergraph is presented in Figure~\ref{fig:hypergraph}.

%       If we were to simplify our hypergraph to the size of the example, then Figure~\ref{fig:hypergraph} would be our hypergraph with seven nodes $u \in V$ and five hyperedges
% $e \in E$. These nodes would represent vehicles and each hyperedge represents a specific attribute as we can see in the legend.

%       It should be noted that in the hypergraph we used in our experiments, there were several more hyperedges.

Organizing the information relating to specific agent attributes using hyperedges, enables us to both access this information efficiently, and keep it organized.
Moreover, in many settings, agent characteristics captured by hyperedges, naturally correspond to criteria according to which we can form coalitions.
For example, it is conceivable that we want to use agents with {\em high capacity} from the respective {\em high-capacity} edge, if our goal is to form coalitions with {\em high capacity}. Our approach of using hypergraphs is even more generic than what implied so far, since we can easily define hyperedges that contain agents which are or are not {\em permitted} to connect with each other, for various reasons; and since we can exploit the hypergraph to allow the formation of coalitions according to a multitude of criteria.

\section{Criteria for Forming Coalitions}
\label{subsec:criteria}

The algorithms presented in this work can be employed by any entity or enterprice (such as the Grid, utility companies or Smart Grid cooperatives) that wants to form EV coalitions for the V2G problem, using any set of criteria of its choosing. Here we identify three such natural criteria, namely \textit{reliability, capacity} and \textit{discharge rate}. These formation criteria are consistently mentioned in the related literature, though perhaps not with these exact names, and not explicitly identified as such~\cite{kamboj2010exploring,kamboj2011deploying,valogianni2014effective}.

First of all, a coalition has to be consistently {\em reliable}, i.e. it should be able to provide the power that has been requested without any disruptions. For a coalition to be reliable, its members must be reliable too, and gaps in reliability must be met with backup agents. We define {\em agent reliability} as {\em the estimated probability that an agent will fulfill its promises}. The {\em promise} of an agent is its {\em commitment} on being connected to the Grid during a specific time slot in order to contribute via providing energy to the Grid, if so requested. Such slots naturally correspond to electricity trading intervals.

Since the coalitions are formed to offer power services in future time slots, agents can be asked to state their availability.
This availability is stored in commitment hyperedge .

In addition, a coalition must fulfill a {\em capacity} requirement. 
The {\em capacity} of a coalition is the amount of electricity (measured in $kWh$) the coalition will be offering to the Grid; 
while the capacity of en EV is, similarly, the amount of electricity (in $kWh$) the EV will be offering to the Grid.
In fact, gathering enough EV capacity to cover the Grid needs during high demand periods, is the main objective of any V2G solution. 
Naturally, creating a coalition to meet a high power peak requires a considerable amount of capacity offered. 
On the other hand minor peaks can be stabilised by building EV coalitions with a much lower capacity.

Another factor in the V2G problem is the {\em discharge rate} of a coalition (or, of a single EV)---the rate by which the coalition (resp., the EV) is able to provide (electrical) energy to the Grid over a specified time period. Discharge rate is measured in $kW.$ % (=$kWh / h$).
A high coalitional discharge rate could be required in cases where capacity should be offered within a small amount of time, for example when the Grid is under a heavy demand load. 
Naturally, a coalition has a high discharge rate if its members discharge rates are high; for our purposes, we assume that the discharge rate is additive, i.e., the discharge rate of a coalition is the sum of its EVs discharge rates.
In Chapter~\ref{Chapter7_results}, we will be forming coalitions in order to meet specific capacity and discharge rate targets; and observing how reliable the coalitions meeting these targets are.

Now, the hypergraph used in our current implementation was designed so that it could easily satisfy requests pertaining to these particular criteria.
As such, there was a total of $25$ hyperedges in the hypegraph---\{{\em extremely-high, very-high, high, medium-high, medium-low, low, very-low, extremely-low}\} $\times$ \{{\em capacity, discharge rate, reliability}\}; and a {\em committed} one, containing EVs that have stated they will be connecting to the Grid during the particular slot.\footnote{We could have stored the commitment of the EVs on a ``per time slot'' basis, by using several hyperedges (one per slot) without any additional cost. However, in our experiments, we focus on a single time slot only.}

In our model, we assume that, at any time step that this is required---due to a consumption peak, an unplanned event, or the need to regulate frequency and voltage---the Grid (or some other entity) advertises its demand for a V2G coalition with several desirable characteristics. As noted in~\cite{kamboj2011deploying}, individual EVs are well-suited for providing services at short notice. What we show in this thesis, is that we can select agents from a huge pool of EVs to form {\em coalitions} that are able to provide large amounts of power at short notice, and with high reliability.


%To effectively extract nodes from the hypergraph we put forward the following three methods.
%\begin{itemize}
%	\item {Minimal Transversal} \cite{kavvadias2005efficient}
%		\item {Clustering} \cite{zhou2006learning}
%		\item {Heuristic Approach}
%	\end{itemize}
%	Preceding the above approaches, though, we have noticed that pruning the hypergraph to include EVs only from the hyperedge we are interested in, is extremely efficient. For example, we always prune the hypergraph to keep nodes from the "Committed" hyperedge, and we also prune the hyperedges that signify low values of an attribute.


\section{Pruning the Hypergraph}\label{sec:pruning}

An important aspect of using hypergraphs for dealing with large state-spaces, is the resulting ability to perform node and edge pruning. Since dozens or hundreds of thousands of our EVs populate the hypergraph, and each one is a member of several hyperedges, running the algorithms without pruning would require an enormous amount of computing power. However, due to the nature of the hypergraph, and the way we store our vehicles and their attributes, it is extremely easy and effective to narrow down the number of vehicles and edges used, by leaving out EVs that are less promising as coalition members. For example, if achieving a high capacity for the to-be-formed coalition is a key goal, then, intuitively, we can narrow down our search for coalition members by focusing only on nodes belonging to the set of hyperedges (or ``categories'') $high capacity \cup very high capacity \cup ex high capacity$. 

% We use this method in all our algorithms with small modifications to each one, in order to relieve bottlenecks.	
To illustrate pruning, Fig.~\ref{fig:hypergraph} shows a hypergraph that contains all EVs. In order to reduce the size of the hypergraph and thus the computing requirements, we could keep only EVs belonging to at least one high quality edge, as shown in Fig.~\ref{fig:hypergraph_prune}. 


\begin{algorithm}
	\caption{Pruning the Hypergraph}\label{alg:pruning}
	\begin{algorithmic}[1]
		\Procedure{Pruning}{$H$, $CategoriesKept$}
		\For{Hyperedge $\in$ H}
		\If{$Hyperedge \in CategoriesKept\cap Committed$}
		\State $NewHEdges\gets NewHEdges \cup HyperEdge$
		\State $NewNodes\gets NewNodes \cup HyperEdge.nodes$
		\EndIf
		\EndFor
		\State $NewHGraph \gets Hypergraph(NewNodes, NewHEdges)$
		\EndProcedure
	\end{algorithmic}
\end{algorithm}

Algorithm~\ref{alg:pruning} is our implementation of pruning. The algorithm iterates over all hyperedges in the given hypergraph $H$, and 
keeps only the nodes belonging to hyperedges that correspond to the specified ``categories of interest'' ({\em CategoriesKept} in Alg.~\ref{alg:pruning}).

In our implementation, the {\em CategoriesKept} are heuristically selected, and depend on the algorithms. For instance, the {\em minimal transversal} algorithm requires a more aggressive pruning, since its complexity is sensitive to the number of nodes used as input (cf. Section~\ref{sec:transversal}), and we therefore empirically feed it with as few hyperedges as possible. 
In section~\ref{sec:generating} we provide Table~\ref{tab:pruningres} showing the efficiency of our pruning algorithm.

Our experimentation indicates that the use of pruning can lead to a significantly smaller hypergraph, and to vast improvements in terms of execution time for our algorithms.  In our simulations, the hypergraphs are pruned to about $1/20$ of the initial size of the EVs pool, without sacrificing the methods' performance (cf. Section~\ref{sec:generating}). Moreover, pruning using Algorithm~\ref{alg:pruning} is almost instantaneous.

\section{A Minimal Transversal Algorithm} \label{sec:transversal}
Using hypergraphs allows to use an intuitive approach  for locating agents for coalitions: to generate the set of \textit{minimal transversals} for the \textit{high-value hyperedges}~\cite{eiter1995identifying}. A \textit{transversal} (or \textit{hitting set}) of a hypergraph H, is a set $T\subseteq V$ with hyperedges $X$ where $X = E$ (i.e., vertices in $T$ belong to \textit{all} hyperedges in $E$). A \textit{minimal transversal} is a set that does not contain a subset that is a hitting set of $H$.  As such\footnote{Of course there can be more than one minimal transversals, and it is not necessary that they have the same cardinality.}, generating several minimal transversal sets for \textit{high-quality} hyperedges is expected to identify agents which are high-quality and should be used in the formation of a coalition. Subsequently, we join those agents together until our criteria are met. 

Our approach with the minimal transversal set is to prune all edges but those of extremely high quality that are also ``committed'', as seen in Algorithm~\ref{alg:transversal}. Then we generate progressively the minimal hitting sets, using an algorithm similar to \cite{eiter1995identifying}. That is, we first generate the minimal hitting sets containing one node, then those with two, and so on. Then we randomly pick agents belonging to those minimal transversals, until the coalitions requirements are met. If the requirements are met during the progressive minimal transversal generation process, no further minimal transversals are generated.

To illustrate this concept with the help of Fig.~\ref{fig:hypergraph}, we prune the hypergraph to keep only the high-quality edges $e_1, e_3, e_5$, leaving us with the nodes $u_1, u_3...u_7$ and edges $e_1, e_3, e_4$, as seen in Fig.~\ref{fig:hypergraph_prune}. Then we generate all the minimal transversal sets. The minimal transversals generated first are the ones with two nodes (since there are no minimal transversals with one node) i.e. the following$\{u_3, u_5\}, \{u_1,u_7\}, \{ u_6, u_1\}$.%,\newline$\{ u_3, u_7\},\{ u_3, u_6\} $.

This method creates a set of agents with uniformly distributed high-quality characteristics. Though this is desirable in theory, in practice the results vary depending on the generated minimal transversal set. There are characteristics which might be of higher importance than others and this cannot be taken into account by the transversal algorithm due to its nature. Regardless, this method could be of much use for creating a base of quality agents; for uniformly improving the quality of an already formed coalition by adding agents from the minimal transversal sets; and for creating versatile coalitions without focusing on specific attributes.

\begin{algorithm}
	\caption{Coalition formation using Minimal Transversal}\label{alg:transversal}
	\begin{algorithmic}[1]
		\Procedure{MinimalTransversal}{$H$}
		\State $H \gets Prune(H, exhigh)$ \Comment exhigh signifies all hyperedges with exhigh qualities
		\State $T= \emptyset$, $C = \emptyset$ \Comment Start with an empty coalition
		\For{i=1 to $|E|$} \Comment where $|E|$ is the number of edges in the (pruned) $H$
		\State Create the union $U$ of minimal transversal sets with size $i$, generated from $H$.
		\State $T$ = $T$ $\cup$ $U$
		\While{$C$ does not meet the criteria}
		\State Randomly select an {\em unselected} node  $\in T $ and add it to $C$
		\EndWhile
		\If{criteria have been met}
		\State return formed coalition $C$
		\EndIf
		\EndFor
		\EndProcedure
	\end{algorithmic}
\end{algorithm}

Line 6 of Algorithm~\ref{alg:transversal} is our implementation of minimal transversal~\cite{eiter1995identifying}. Though there is no known polynomial time algorithm for the general hypergraph transversal problem, the algorithm given was shown experimentally to behave well in practice, and its memory requirements are polynomially bounded by the size of the input hypergraph, though it comes without bounds to its running time.



\section{A Clustering Algorithm}\label{sec:Clustering}
The second approach is to create clusters of agents. After creating said clusters, we efficiently calculate the best cluster and then sample EVs from that group until our coalition criteria are met.

In more detail, we first generate a hypergraph of EV agents with the characteristics described previously. Then, hypergraph clustering is performed.
The hypergraph clustering itself is an implementation of that proposed in \cite{zhou2006learning}, and is conducted as follows. 	%On the hypergraph we have generated we also add weights in the hyperedges. 
%The weights correspond on the label (high, medium, low) of the attribute and are going to be used later for locating high valued agents.

We begin by implementing functions that calculate
\begin{itemize}
	\item{\em the Incidence Matrix}: A %$|V|\times|E|$%
	matrix $H$ with entries $h(u,e) = 1$ if $u \in e$ and $0$ otherwise.
	\item{\em the Weight Matrix}: A diagonal matrix $W$ containing the weights of the hyperedges.
	\item{\em $D_u$ and $D_e$}: Matrices containing the node and hyperedge degrees respectively.
	\item{\em the Adjacency Matrix}: A matrix defined as $A = HWH^T - D_u$
\end{itemize}
The matrices above are used for the final calculations of the hypergraph \textit{Laplacian matrix}. This a matrix representation of a graph, that has information on the degrees of the nodes, and their connections with the hyperedges (cf.~\cite{zhou2006learning}, Section 5). 
After its calculation, the Laplacian contains the node degrees in its diagonal (which enables us to discard the $D_u$ matrix, for memory efficiency).

As explained in~\cite{zhou2006learning}, having the Laplacian, enables us to calculate the $\Phi$ eigenvectors $[\Phi_1 ... \Phi_k]$ corresponding to the $k$ lowest eigenvalues. These can then define $X = [\Phi_1 ... \Phi_k]$, a matrix that can be employed for $k$-way partitioning to cluster our agents. This is achieved via running the $k$-{\em means} algorithm \cite{hartigan1979algorithm} on the row vectors of $X$\cite{zhou2006learning}. As explained in \cite{zhou2006learning}, the rows of X are representations of the hypergraph vertices in the $k$-dimensional Euclidean space. Of course, choosing a value for $k$ has to be decided empirically. In Section~\ref{sec:results_modifications} we will be testing different values for $k$. 
After generating the clusters, we are given the task to locate the ``best'' cluster among them. To do this efficiently, we simply sort them by looking at {\em the average of the node degrees}.
\footnote{Note that the Laplacian matrix can also be used to extract easily high-quality agents, by retrieving nodes that have high values (high node degrees) in its diagonal.}
This provides us with a cluster that is better than the rest. We then sample nodes from the best cluster until our criteria are met. Algorithm~\ref{alg:clustering} summarizes the method.

\begin{algorithm}
	\caption{Coalition formation using Hypergraph Clustering}\label{alg:clustering}
	\begin{algorithmic}[1]
		\Procedure{Clustering}{$H$}
		\State $H \gets Prune(H, (vhigh \cup exhigh))$ \Comment exhigh and vhigh signify the sets of extremely high and very high quality hyperedges respectively
		\State Generate k clusters using the algorithm described in~\ref{sec:Clustering}~\cite{zhou2006learning}
		\State $C = \emptyset$ \Comment Start with an empty coalition
		\State Find the best cluster, $A$, by comparing the sum of node degrees of each cluster.
		\While{$C$ does not meet the criteria}
		\State Randomly select a node $\in A $ and add it in $C$
		\EndWhile        
		
		\EndProcedure
	\end{algorithmic}
\end{algorithm}

\section{A Heuristic Algorithm} \label{sec:heuristic}
While using a minimal transversal generates quality sets of agents, computing the {\em degree} of a node can identify single agents with many quality attributes. As an example, when we have a reliable coalition as a base but we require more capacity, we can use the sorted list we have generated, to pick agents with high capacity. Intuitively, this approach will result to picking high overall quality agents for our coalition. We can also create coalitions by using only the best available agents. Moreover, we can use the aforementioned sorted-by-degree list of nodes in order to "fill gaps" and improve on the quality of already formed coalitions. 

Thus, our heuristic method operates as follows. {\em (1)} First, we prune the hypergraph to include only ``promising'' nodes and hyperedges. For instance, we exclude nodes not in {\em extremely high} or in {\em very high} hyperedges. {\em (2)} Then we sort the remaining nodes based on their node degree. {\em (3)}  Finally, we pick the highest degree nodes from the list until the coalition criteria are met. By starting at the top of the list, we can guarantee that agents have many positive characteristics. 

We can see at step {\em (1)} above, that this algorithm, like the rest of our methods, employs pruning. As such, it does exploit the hypergraph structure. However, in practice the algorithm can deliver excellent results without much pruning. In our experiments in Chapter~\ref{Chapter7_results} below, the heuristic approach is shown to outperform the rest while pruning only the non-committed nodes in the hypergraph. In fact, one strength of this approach is that it does not {\em rely} on pruning, since its complexity is low: essentially, that of the algorithm employed for sorting (i.e., $O(nlogn)$, since we use with Python's built-in {\em Timsort} algorithm). By not relying on pruning, the algorithm can focus on promising nodes with high node degree (and, therefore, quality), irrespective of the exact hyperedges to which they belong.

\section{A Hybrid Algorithm} \label{sec:hybrid}

In an attempt to exploit the strengths of each method we devised a hybrid algorithm that selected agents using both the transversal and heuristic method. As mentioned above, the transversal algorithm has the ability to find coalitions that are good in all aspects. The heuristic algorithm though, identifies single agents that are good depending on the weight of the edges.

The hybrid method works as follows. {\em (1)}It prunes the hypergraph using the methods described to shrink the size of the pool. {\em (2)}Then the transversal algorithm runs and generates a coalition with $k$ times the needed goals. The agents of the new coalition are $V_1 \subset V$.  {\em (3)} Finally, instead of randomly selecting agents from the minimal transversals generated, the heuristic method runs on a hypergraph $H_1 = (V_1, E)$ and selects the final coalition. It will also fill in the rest of the gaps in the coalition  if required. This is better shown in~\ref{alg:hybrid}.

Thus, the hybrid method can guarantee that coalitions generated share the properties of coalitions formed through the simple transversal algorithm, but also manages a better quality by selecting the best agents using the heuristic method. 

\begin{algorithm}
	\caption{Coalition formation using Hybrid Approach}\label{alg:hybrid}
	\begin{algorithmic}[1]
		\Procedure{Hybrid}{$H$}
		\State $H \gets Prune(H, (vhigh \cup exhigh))$ \Comment exhigh and vhigh signify the sets of extremely high and very high quality hyperedges respectively
		\State Generate a coalition $V_1 \subset V$ using algorithm~\ref{alg:transversal} but using $k$ times the requirements (goals) needed. (k is empirically selected)
		\State Generate a hypergraph $H = (V_1, E)$ and run the heuristic algorithm for possibly the final coalition.
		\State If needed (k < 1), run heuristic again to fill in the gaps. 
		
		\EndProcedure
	\end{algorithmic}
\end{algorithm}

This method does have a disadvantage. While the resulting coalitions are of high quality, the runtime is much higher than the simple methods since both algorithms run sequentially (with a lower goal, though). Nevertheless, the runtime is still low and can be further reduced using methods discussed in Chapter~\ref{Chapter8_results}.


\section{A Simple Sampling Method}
For interest, and in order to have a benchmark for the rest of our algorithms, a simple sampling algorithm was also developed. The algorithm takes random samples until the specified goals are achieved.
%We will implement the methods above and compare the results.
\include{Chapters/Chapter7_results}
 % Chapter 1

\chapter{Conclusions and Future Work} % Main chapter title

\label{Chapter8_results} % For referencing the chapter elsewhere, use \ref{Chapter1} 

	In this thesis, we demonstrated how to employ hypergraphs for creating coalitions based on multiple criteria.  
The existence of several hypergraph transversal and clustering algorithms
makes hypergraphs easy to work with. Moreover, the ability to select almost instantaneously parts of the hypergraph that are interesting, offers a significant advantage, 
enabling one to generate coalitions with desirable characteristics within seconds. This makes hypergraph use quite attractive for real-world, real-time scenarios.

We presented several coalition formation methods that employ hypergraphs for tackling the V2G problem, and evaluated their performance. 
Our proposed heuristic algorithm, in particular, was shown to be the most effective and efficient of our methods, 
as it is able to use a minimal number of EVs to provide the required capacity, discharge rate, and reliability to the Grid in a few milliseconds; while it exhibits exceptional scaling behaviour with respect to the number of EVs under consideration. Ours is the first approach that is able to deal with {\em large-scale} coalition formation for the V2G problem, while taking {\em multiple criteria} into account for creating the EV coalitions.

Future work includes implementing a more efficient {\em minimal transversal} algorithm as follows.

Finding all the minimal transversals of a hypergraph is a computationally difficult task. As the size of the graph increases the number of patterns increases - exponentially in the worst case scenario. Nevertheless, while the pool of EVs might be huge, a coalition meeting the requirements could be small enough and require only a handful of transversals to be generated. 

For this reason, it is worth exploring ways to generate transversals in a depth-first manner. This would enable us to create as many as we required to fulfill the requirements of the coalition. Such an attempt was presented in~\cite{kavvadias2005efficient}. This paper suggest a way to create transversals one by one as opposed to generating simultaneously all sets of each size, and while it doesn't offer a time bound for the worst case scenario, it does offer bound in terms of memory use.

By implementing this algorithm, we could stop the execution as soon as the coalition being built reached the requirements. This can greatly the execution time of the Minimal Transversal algorithm and allow it to scale to higher pool sizes. This method, since it is a depth-first approach, might even move the scaling bottleneck from the size of the EV pool to the requirements.

Another algorithm, proposed in~\cite{khachiyan2005new}, for finding $k$ minimal transversals using parallel processors. The time is bound by $\text{polylog}(|V|, |H|, k)$ assuming $\text{poly}(|V|, |H|, k)$ numbers of processors. This algorithm could be the most efficient for finding a collection of minimal transversals but unfortunately an estimation for $k$ must be made first.

By implementing this algorithm we could gain several advantages. For instance the algorithm would not scale with pool size anymore but only with requirements. It could eventually be faster than heuristic since it can run in multiple processors (sorting used in heuristic cannot run in multiple cores). Nevertheless, the results should generally stay the same (coalition size for example) since it is only an optimization.
 
Finally, additional future work includes improving the clustering approach with an alternative method for representing the vertices in the Euclidean space; and for identifying promising clusters.	Finally, all algorithms can be equipped with multithreading capabilities, to substantially improve their performance.



%----------------------------------------------------------------------------------------
%	THESIS CONTENT - APPENDICES
%----------------------------------------------------------------------------------------

%\appendix % Cue to tell LaTeX that the following "chapters" are Appendices

% Include the appendices of the thesis as separate files from the Appendices folder
% Uncomment the lines as you write the Appendices

%\include{Appendices/AppendixA}
%\include{Appendices/AppendixB}
%\include{Appendices/AppendixC}

%----------------------------------------------------------------------------------------
%	BIBLIOGRAPHY
%----------------------------------------------------------------------------------------

\printbibliography[heading=bibintoc]

%----------------------------------------------------------------------------------------

\end{document}  
