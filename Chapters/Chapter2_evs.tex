% Chapter 1
\chapter{Background}
\label{Chapter2_bg}
This chapter presents some background for the research presented in this thesis. Specifically, section~\ref{Chapter2_evs} starts with a definition and overview of the Smart Grid and connects it to electric vehicles. In section~\ref{Chapter3_cf} we discuss coalition formation and finally, in section~\ref{Chapter4_hg} we present background information on hypergraphs.

\section{Electric Vehicles in the Smart Grid} % Main chapter title

\label{Chapter2_evs} % For referencing the chapter elsewhere, use \ref{Chapter1} 
%\label{sec:introduction}
%----------------------------------------------------------------------------------------

%----------------------------------------------------------------------------------------

The current electricity Grid, is the network that delivers power to consumers. It uses large central power stations that distribute the energy through high capacity power lines to both industrial and domestic areas. Historically the Grid handled peak hours poorly, with blackouts and power cuts being common~\cite{gorowitz2000general}. Only more recently and after establishing patterns in electricity demands, could the daily peaks be met, using part-time generators (usually the expensive gas turbines). The current structure of the Grid is a product of an evolutionary process that lasted decades, connected to growing needs of consumers. Thus, the current infrastructure was not planned as a whole, but rather extended several times, creating weak links and containing outdated designs. One of the most important issues, is the centralized nature of the Grid. It is built around large power plants producing all the energy required by the consumers. With the growth of smaller, usually renewable, energy producers, the Grid by necessity has to move to a less centralized and more interactive structure.

The Smart Grid, therefore, is a modernized electricity Grid that collects and uses information to improve efficiency, reliability, economics and sustainability of the electrical Grid. It is also planned to be more decentralized making efficient use of small scale producers and prosumers.~\footnote{Prosumer is a small scale electricity consumer that might also produce energy\cite{lampropoulos2010methodology}.} Thus, the Smart Grid can potentially become much more reliable than its "classic" counterpart by eliminating single points of failure. Another important characteristic of the Smart Grid is that it is much more interactive. Everyone connected, communicates and coordinates with the Grid, rendering the consumption and production more balanced. Connected consumers also implement smart technologies that drive their own consumption down. With techniques like the ones mentioned above, the network's energy load is balanced and thus the distribution is more efficient - eliminating, if possible, high-cost energy producers like gas turbines. To understand the efficiency of the Smart Grid, experiments have, for example, shown that just a small scale coordination of a few battery-equipped houses can lower everyone's electric bill~\cite{vytelingum2010agent}.
%\section{Electric Vehicles}

Electric vehicles (EVs) are a promising new concept for the automotive industry. EVs use energy stored in a battery and electric motors to generate propulsion. Electricity offers many advantages against petrol-powered vehicles. Specifically, EVs are cost effective, require less maintenance, and have no direct emissions since they run in electricity powered engines. In their current state, the batteries of electric vehicles (which rapidly become even more efficient and cost effective~\cite{nykvist2015rapidly}) are capable of at least 300km~\cite{globalev2016}\cite{young2013electric} of range.~\footnote{The range of an EV is defined as the driving range using power only from its battery pack during a single charge.} To achieve this range, the batteries have a large capacity usually in the 60kWh-100kWh range. To put this into perspective, batteries as low as 4kWh can have a significant impact on the energy footprint of a typical household~\cite{vytelingum2010agent}.

Another important factor for the battery is the discharge rate. A vehicle requires a large amount of energy during acceleration. For example, accelerating a typical vehicle to 100km/h in 10 seconds can require up to 65kW of power~\cite{young2013electric}. Since EV batteries are actually designed for discharging at these rates for typical driving, we can safely assume that we can use this discharge rate for other uses.

Charging the batteries is another characteristic that must be accounted for. Currently, charging the battery takes a few hours depending on the battery's state of charge (Soc). Nevertheless for an everyday use scenario, a battery can be expected to charge (using fast charging) to a reasonable amount in half an hour~\cite{young2013electric}. In this thesis we will not examine how EVs charge, or how we can regulate its charging.

Due to the previously mentioned growth of EVs and their energy capacity we can safely assume that they will play a significant role in the future of the electricity Grid~\cite{ramchurn2012putting}. As a result, two categories of problems arise. First, the issue of how we can successfully provide the energy those vehicles need, and charge them without overloading the Grid. The second is how energy stored in EVs can be used to balance out peaks in consumption or even serve as backup power. Those categories are called Grid to Vehicle (G2V) and Vehicle to Grid (V2G)~\cite{loisel2014large}~\cite{kempton2005vehicle} respectively.

G2V is better explained by noticing that, due to the common working hours, large numbers of EV owners might return home and charge their vehicles, at about the same time. Since EVs can draw a huge amount of power the Grid will overload due to huge spikes on consumption. Nevertheless the charging could have been coordinated and EVs charged during the night, without causing spikes. Finding an optimal way to charge EVs this way though is quite complicated, since possibly millions of batteries have to eventually be charged. Several attempts have been made to tackle the problem~\cite{gan2013optimal}\cite{valogianni2014effective} but are not usually scalable to large numbers of EVs.

V2G, a problem related to our work here, in contrast to G2V, is the question of how EVs can supply power (usually stored in the vehicle's battery) to the Smart Grid during power peaks. This can lower or even eliminate the need of expensive back up generators. Since the batteries can charge when power is cheap (e.g. at night) and return the power when its more expensive, this raises an opportunity for profit for EV owners. Nevertheless this does not come without several issues to be addresses. Specifically, a single EV must know if its owner will need the energy that will be sent to the Grid and not participate in an exchange if there's a chance the owner needs to use the vehicle. In addition, EVs can be used by owners without any previous notice and might be unplugged from the electricity Grid at any moment. This raises the issue of reliability: how certain we are that a vehicle that promises to deliver power during a timeslot, can actually keep its promise. Finally, while the batteries can store and provide a respectable amount of energy, the needs of the Grid are proportionally much greater. Thus EVs must be able to cooperate to provide sufficient and reliable services. Several aspects of V2G have been researched. We will mention several such attempts in Chapter~\ref{Chapter5_related}.


%Electric vehicles (EVs) are a promising new concept for the automotive industry. EVs use energy stored in a battery and electric motors to generate propulsion. Electricity offers many advantages %against petrol-powered vehicles. Specifically, EVs are cost effective and require less maintenance, and thus have no emissions since they run in electricity powered engines. The growing popularity of EVs %gives rise to the so-called G2V and V2G problems. G2V describes a system where EVs connect and draw power from the Grid without overloading it\cite{valogianni2014effective}. V2G is the problem of EVs %communicating with the Grid in order to either lower their power demands or return power back to the network when there is a peak in the request for power. This helps the Grid to maintain a balanced power %load\cite{ramchurn2012putting}. This is the problem we will be dealing with in this paper.

%	An important issue in the V2G problem is that there are possibly millions of EVs which communicate and connect to the Grid. The vast number of vehicles means that we must create the most appropriate %groups to cover the needs of the Grid at any given time. Algorithms that scale well and give results almost instantly are necessary. 

%	In order to tackle the V2G problem, we resort to coalition formation. Specifically, we propose the formation of coalitions using hypergraphs. By doing so, we can efficiently locate reliable agents and form %effective EV cooperatives to provide sufficient energy and stability. 

%	Such attempts use mostly machine learning or attempting to form the optimal coalition\cite{deORamos2014}\cite{valogianni2014effective}. This had the drawback that it did not scale to more than a few %hundred agents \cite{kamboj2010exploring} \cite{deORamos2014}. Besides, the approaches that have been used do not deal with multi-criteria optimization. This is important, however because in reality %coalitions have to be formed according to several criteria such as capacity and discharge rate. In our attempt, we will try to form coalitions by selecting vehicles from a huge pool of individual EVs using 
% multiple criteria for our selections. 


%	The Grid should be able to advertise the amount of power it requires by both asking for a required capacity and a maximum discharge rate. What we are trying to do is fulfill the required capacity and %discharge rate with the minimum amount of vehicles and by keeping our coalition reliable. We are not searching for an optimal coalition but rather for one that can be generated quickly and reliably. We do %this by organizing our electric vehicles inside a hypergraph. Current research and solutions on the V2G problem do not scale well. It should be noted that it is also the first attempt to use hypergraphs for %coalition generation. Hypergraphs are well studied, and powerful algorithms do exist for traversing and exploring them. 

%	In a few words, we start with a huge pool of EVs. We know their power capacity, discharge rate and if they are committed to connect to the Grid. We also know their reliability. The Grid advertises the demand of a coalition with a specific capacity and discharge rate. We form a coalition that fulfills the power requirements and has a high reliability while also being small in size. 
%	Now in order to build coalitions for the V2G we need to combine the capabilities of EVs. This naturally gives rise to a multi-criterion selection problem for choosing the members of a coalition. In order to tackle this problem, we propose a novel, principled approach in order to form coalitions that have specific characteristics. For this we employ the use of hypergraphs and research that has been done on them \cite{zhou2006learning} \cite{kavvadias2005efficient}

%	In general the related work\cite{kamboj2010exploring}\cite{kamboj2011deploying}\cite{deORamos2014}\cite{valogianni2014effective} focuses in single-criteria coalition formation and in near-optimal %solutions that require great processing power and scale poorly. 

%  ***** AYTO as to exoume ypopsin gia pi8anes erwthseis ****
%First of all, while multiple criteria can usually be expressed as a single one with the help of a utility function, we find that multi-criteria is a more natural way to express the agent's attributes. As such we do not use directly utility functions.

%	To continue while finding the optimal coalition is useful in most cases it can't work in real world situations where there could be millions of EVs, and the requirements could be updated every few seconds. Therefore, we sacrifice the ability to find the optimal solution so that we can process thousands of EV's in a few seconds.