% Chapter 1

\chapter{Related Work} % Main chapter title

\label{Chapter5_related} % For referencing the chapter elsewhere, use \ref{Chapter1} 

%----------------------------------------------------------------------------------------

% Define some commands to keep the formatting separated from the content 


%----------------------------------------------------------------------------------------

%\label{sec:related}
Here we review related work mainly on the V2G and the SCG problems, and highlight its differences to our approach in this thesis.
To begin, in their pioneer work, Valogianni {\em et al.}~\cite{valogianni2014effective} propose an {\em adaptive smart charging algorithm} that adjusts {\em the power drawn} from the electrical Grid for charging EVs, based on each EV owner's utility from charging. backbone of the approach is based on a {\em reinforcement learning} for capturing agent needs and behavior. It also utilizes an optimization module that schedules the charging of each EV in order to maximize its utility, subject to network constraints. Though effective, this work fails to focus on the problem of feeding the network with power drawn from EVs in a coordinated fashion. As such EV coalitions and their potential are not being considered in this work.

Contrary to the aforementioned line of research, the work presented in~\cite{kamboj2010exploring} considers an attempt to exploit EV coalition formation in energy exchange. In particular in~\cite{kamboj2010exploring} EV coalitions are utilized in selling power in the regulation market. In more detail, EV coalitions provide the following service to the Grid every few seconds: they, either, (i) {\em scale down} their power draw (or discharge); or they (ii)  {\em scale it up}, and request more power from the Smart electrical Grid.
Despite the effectiveness of this approach, there are considerable limitations with respect to its practical application that renders its usability in real setting scenario, questionable. In more detail there is a need for a complicated and resource-consuming EV selection process by an aggregator agent. Moreover, in this context, and to limit the respective complexity, the simulations involved in this work considered a limited pool of three hundred vehicles only.

That said, the potential of coalition formation is not only exploited in the narrow context of EVs. In more detail, coalition formation has long been investigated to provide regulation services to the Smart Grid and, in recent years several works in this direction emerged. For instance the work in~\cite{vinyals2012stable} adapts a game-theoretic perspective on the formation of coalitions in the Smart Grid. In this context, it considers the optimal coalition structure generation problem (CSG). To this end, it utilizes an approach of forming {\em virtual energy consumer} coalitions. In the context of these coalitions, it manages to flatten the energy demand. This, in turn, enhances the negotiational ability with the Grid, enabling better prices in what could be a G2V arrangement. In more detail, the solution of the CSG, provides the best VEC for every consumer on the market; and guarantees a core-stable payoff distribution outcome. Nevertheless, the computational complexity of this approach renders it impractical in real settings. In particular, this work has been shown to perform adequately on social graphs of limited size (with only a handful of agents). Notably, against this background, our approach manages to produce high quality solutions in milliseconds, and scales to the number of millions (as further discussed in Chapter~\ref{Chapter7_results}).

Now, two recent papers which study {\em cooperative games} defined {\em over graphs} that impose constraints on the formation of the coalitions, are~\cite{chalkiadakis2016characteristic} and~\cite{chalkiadakis2012coalitional}. Specifically, they assume that the environment possesses some structure that forbids the creation of individual coalitions, due to limited resources and existing physical or even legal barriers. This is captured by an undirected graph providing a path connecting any two agents that can belong to the same coalition.
%Communication, transportation, or sensor networks provide natural settings for cooperative games over graphs.
Both of these papers, however, do not employ hypergraphs in any way. Hypergraphs have in fact been used for modelling agent interactions in cooperative game settings, where agents can simultaneously belong to multiple coalitions~\cite{jun2009hypergraph} \cite{zick2012overlapping}. Now, several papers~\cite{chalkiadakis2016characteristic}~\cite{chalkiadakis2012coalitional}~\cite{jun2009hypergraph}~\cite{zick2012overlapping} focus on studying the theoretical problem of achieving {\em coalitional stability} via appropriately distributing the payoff among the agents. This is done rather than providing algorithms for large-scale coalition formation in real-world settings, as we do in this work. 

By contrast, two papers that study the generation of optimal coalition structures while focusing on stability are~\cite{bistaffa2014anytime}~\cite{voice2012coalition}. They focus on the use of synergy graphs. Those graphs connect agents with edges that represent a vital synergistic link, such as communication, trust or physical constrains. They propose efficient ways to generate all possible coalitions and find the optimal coalition structure.  Although these approaches scale to thousands of agents they are limited in terms of scalability compared to our approach which scales to millions of users. Furthermore, their approach fails to tackle multiple formation criteria.

A paper that is more related to our work here, in the sense that it exploits constraints among vehicles for coalition formation, is the work of Ramos {\em et al.}~\cite{deORamos2014}. In this context, they propose the dynamic formation of coalitions among EVs so that they can function as {\em virtual power plants} that sell power to the Grid as an aggregate. The method relies heavily on a inter-agent negotiations protocol. However, that work also attempts to tackle the optimal CSG problem and hence suffers from high complexity and scalability issues. As such, although it is empirically shown to produce solutions that are close to optimal (98\%), this is only when tested in scenarios with a few dozens of agents. In addition, the work in~\cite{deORamos2014} considers only a single criterion for the formation of a coalition---namely, the {\em physical distance} among the EVs. The physical distance, however, is not a very natural criterion; and, in any case, it is imperative that a multitude of criteria is taken into account---such as capacity, discharge power, and perceived reliability (see, e.g.,~\cite{kamboj2011deploying}). Our approach, in contrast, is able to take into account any number of natural criteria to form EV coalitions.