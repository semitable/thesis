% Chapter 1

\chapter{Related Work} % Main chapter title

\label{Chapter2} % For referencing the chapter elsewhere, use \ref{Chapter1} 

%----------------------------------------------------------------------------------------

% Define some commands to keep the formatting separated from the content 


%----------------------------------------------------------------------------------------

%\label{sec:related}

Here we review related work, and highlight its differences to our approach.
To begin, Valogianni {\em et al.}~\cite{valogianni2014effective} propose an {\em adaptive smart charging algorithm} that adjusts {\em the power drawn} from the Grid for charging EVs, based on each EV owner's utility from charging. The approach employs {\em reinforcement learning} for capturing agent needs and behaviour; and an optimization module schedules the charging of each EV to maximise its utility subject to network constraints. Though effective, it does not focus on the problem of feeding the network with power drawn from EVs in a coordinated fashion, and as such there is no mentioning of EV coalitions in that work.

By contrast, an attempt to explicitly sell power in the regulation market via the formation of EV coalitions is presented in~\cite{kamboj2010exploring}. In that work, EV coalitions provide the following service to the Grid every few seconds: they either {\em scale down} their power draw (or discharge); or they {\em scale it up}, and request more power.
The approach is quite effective, but there is a need for a complicated EV selection process by an aggregator agent,
and simulations presented in that paper involved a pool of 300 vehicles only.

A paper adopting a game-theoretic perspective on the formation of coalitions in the Smart Grid is~\cite{vinyals2012stable}. It constitutes an attempt to solve the optimal coalition structure generation problem (CSG). Forming {\em virtual energy consumer} coalitions, manages to flatten the demand in order to get better prices in what could be a G2V arrangement. By solving the CSG, it finds the best VEC for every consumer on the market; and guarantees a core-stable payoff distribution outcome. Unfortunately, this solution is shown to work on social graphs with only a handful of agents. By contrast, our approach manages to produce high quality solutions in milliseconds, and scales to the millions.

Two recent papers which study {\em cooperative games} defined {\em over graphs} that impose constraints on the formation of the coalitions, are~\cite{chalkiadakis2016characteristic} and~\cite{chalkiadakis2012coalitional}. Specifically, they assume that the environment possesses some structure that forbids the creation of individual coalitions, due to limited resources and existing physical or even legal barriers. This is captured by an undirected graph providing a path connecting any two agents that can belong to the same coalition.
%Communication, transportation, or sensor networks provide natural settings for cooperative games over graphs.
Both of these papers, however, do not employ hypergraphs in any way. Hypergraphs have in fact been used for modelling agent interactions in cooperative game settings where agents can simultaneously belong to multiple coalitions~\cite{jun2009hypergraph} \cite{zick2012overlapping}. Nevertheless, all of these papers~\cite{chalkiadakis2016characteristic}~\cite{chalkiadakis2012coalitional}~\cite{jun2009hypergraph}~\cite{zick2012overlapping} focus on studying the theoretical problem of achieving {\em coalitional stability} via appropriately distributing the payoff among the agents; rather than providing algorithms for large-scale coalition formation in real-world settings, as we do in this work. By contrast, two papers that study the generation of optimal coalition structures while focusing on stability are~\cite{bistaffa2014anytime}~\cite{voice2012coalition}. These approaches scale to thousands of agents - but not to millions, as ours (which does not form optimal coalitions), and do not tackle multiple formation criteria.

A paper that is more related to our work here, in the sense that it exploits constraints among vehicles for coalition formation, is the work of Ramos {\em et al.}~\cite{deORamos2014}. They propose the dynamic formation of coalitions among EVs so that they can function as {\em virtual power plants} that sell power to the Grid. However, that work also attempts to tackle the optimal CSG problem. The method relies heavily on a inter-agent negotiations protocol; and is empirically shown to produce solutions that are close to optimal (98\%), but this is when tested in scenarios with a few dozens of agents only. Moreover, there is only a single criterion for the formation of a coalition---namely, the {\em physical distance} among the EVs. Physical distance, however, is not a very natural criterion; and, in any case, it is imperative that a multitude of criteria is taken into account---such as capacity, discharge power, and perceived reliability (see, e.g.,~\cite{kamboj2011deploying}). Our approach, by contrast, is able to take into account any number of natural criteria to form EV coalitions.